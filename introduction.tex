\chapter{Introduction}
\label{chap:introduction}

Field Programmable Gate Arrays (FPGAs) are increasingly being used in embedded and other systems. Such designs often employ one or more embedded microprocessors, and there is a trend to migrate these microprocessors to the FPGA platform primarily for reducing costs. While these soft processors cannot typically match the performance of hard processors, soft processors are flexible allowing designers to implement the exact number of processors desired and to customize them to efficiently fit the application's requirements.

Current commercial soft processors such as Altera's Nios~II~\cite{niosii} and Xilinx's Microblaze~\cite{microblaze} use in-order pipelines with five to six pipeline stages. These processors are often used for less computation-intensive applications such as system control tasks. To support more compute-intensive applications, a key performance improving technique is branch prediction. Without branch prediction, a branch has to execute completely before the processor can fetch the instructions that follow. Branch prediction eliminates these stalls by guessing the target address of branches. Current state-of-the-art branch prediction techniques, e.g., TAGE~\cite{tage}, rely on dynamically collected information about past branch behaviour. Such techniques have been proven to be very effective even in deeply pipelined, highly speculative, high-performance custom processor designs.

Branch prediction has been extensively studied, mostly in the context of application specific custom logic (ASIC) implementations. However, na\"ively porting ASIC-based branch predictors to FPGAs results in slow and/or resource-inefficient implementations as the tradeoffs are different for reconfigurable compared to custom logic. Accordingly, this work studies the FPGA implementation of several commonly used and advanced branch predictor designs and does so in the context of simple pipelined processors, the most commonly used general purpose soft processor architecture due to its excellent balance of performance and resource cost. For this purpose, it assumes a pipelined processor implementation representative of Altera's highest performing soft-processor Nios~II-f and investigates the performance and resource cost of various branch predictors. The analysis confirms that existing designs are not efficient nor high-performing on reconfigurable logic. Accordingly, this work proposes FPGA-specific modifications that improve accuracy, resource cost, or both.

This work proposes branch predictors for Nios~II-f. In more detail, this work makes the following contributions:

(1)~It studies the FPGA-implementation of Branch Target Buffers (BTB), including designs that fuse the BTB and the direction predictor and shows that, contrary to ASIC implementations, it is best to avoid a BTB and instead to calculate branch target addresses on-the-fly.

(2)~It studies the FPGA implementation of the three most commonly used branch direction predictors: bimodal~\cite{bimodal}, gshare, and gselect~\cite{McFarling}. The analysis corroborates the results of past studies showing that gshare achieves the best accuracy among the three for practical table sizes, but also shows that unlike an ASIC implementation, frequency suffers with gshare on FPGAs. It proposes \textit{gRselect}, an FPGA-friendly gselect implementation that uses a simple indexing scheme to outperform gshare by 11.4\%.

(3)~It studies the FPGA implementation of two advanced branch predictors: the Perceptron~\cite{perceptron} and TAGE~\cite{tage} predictors. It optimizes Perceptron's maximum operating frequency by introducing (i) a complement weight table to simplify the multiplication that is otherwise necessary at prediction time, and (ii) Low Order Bit (LOB) Elimination for faster summation. It finds that TAGE is too slow for single-cycle access which negates its advantage in accuracy. Accordingly, this work proposes an overriding predictor \mbox{O-TAGE-SC} that uses a simple base predictor to provide an initial prediction in the first cycle which can be overridden in the second cycle should TAGE disagree with relatively high confidence. \mbox{O-TAGE-SC} achieves 5.2\% better instruction throughput over gRselect.

The rest of the thesis is organized as follows. Chapter~\ref{chap:background} reviews branch prediction basics and introduces the branch prediction schemes considered in this work. It also details the goals of this work. Chapter~\ref{chap:minimal} discusses the design of a minimalistic branch predictor that uses a hardware budget on par with that of Nios~II-f. Chapter~\ref{chap:advanced} presents the study on two more advanced branch predictors: the Perceptron and TAGE predictors. Chapter~\ref{chap:eval} presents the experimental evaluation results and finally Chapter~\ref{chap:conclusion} concludes.





