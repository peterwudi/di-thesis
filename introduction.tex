\chapter{Introduction}
\label{chap:introduction}

Field Programmable Gate Arrays (FPGAs) are increasingly being used in embedded and other systems. Such designs often employ one or more embedded microprocessors, and there is a trend to migrate these microprocessors to the FPGA platform primarily for reducing costs. While these soft processors cannot typically match the performance of hard processors, soft processors are flexible allowing designers to implement the exact number of processors desired and to customize them to efficiently fit the application's requirements.

Current commercial soft processors such as Altera's Nios~II~\cite{niosii} and Xilinx's Microblaze~\cite{microblaze} use in-order pipelines with five to six pipeline stages. These processors are often used for less computation-intensive applications such as system control tasks that are often control flow intensive. To support more compute-intensive applications, a key performance improving technique is branch prediction. Without branch prediction, a branch has to execute before the processor can fetch the instructions that follow. Branch prediction eliminates these stalls by guessing the target address of branches. Current state-of-the-art branch prediction techniques, e.g., TAGE~\cite{tage}, rely on dynamically collected information about past branch behaviour. Such techniques have been proven to be very effective even in deeply pipelined, highly speculative, high-performance custom processor designs.

Branch prediction has been extensively studied, mostly in the context of application specific custom logic (ASIC) implementations. However, na\"ively porting ASIC-based branch predictors to FPGAs results in slow and/or resource-inefficient implementations as the tradeoffs are different for reconfigurable compared to custom logic. Accordingly, this work studies the FPGA implementation of several commonly used and advanced branch predictor designs and does so in the context of simple pipelined processors, the most commonly used general purpose soft processor architecture due to its excellent balance of performance and resource cost. For this purpose, it assumes a pipelined processor implementation that is modeled loosely on Altera's highest performing soft-processor Nios~II-f and investigates the performance and resource cost of various branch predictors. The analysis confirms that existing designs are not efficient nor high-performing on reconfigurable logic. Accordingly, this work proposes FPGA-specific modifications that improve accuracy, resource cost, or both.

We use a two prong approach proposing two classes of predictors that are appropriate depending on the amount of resources we are willing to devote to the predictor component. Branch predictors make use of memories to store past branch prediction outcomes and their accuracy tends to improve the more memory resources they are given. Depending on the device, an FPGA may offer many or very few memory resources. Altera's Nios~II-f uses three Block RAMs (BRAMs) in total and only one BRAM is used for branch prediction~\cite{niosiif}. Therefore, each additional BRAM would represent a more than 1/3 overhead in terms of BRAM resources. Accordingly, we first consider minimalistic branch predictors that also use just one BRAM. In the second approach, we relax this restriction and consider more elaborate and accurate branch predictors and use as many BRAMs as necessary to improve accuracy.

\section{Design Goals}
\label{sec:introduction:goal}

This work aims to design branch predictors that (1)~operate at a high operating frequency while (2)~achieving high accuracy so that they improve execution performance. It proposes a minimalistic branch predictor that has a limited resource budget of one BRAM. The design of the minimalistic predictor considers the most commonly used direction predictors, bimodal~\cite{bimodal}, gshare and gselect~\cite{McFarling} (reviewed in Chapter~\ref{chap:background}). These predictors use a single lookup table and map relatively well onto a single BRAM.

This work also implements more advanced Perceptron and TAGE predictors. As Chapter~\ref{chap:advanced} will show, a single-cycle TAGE is prohibitively slow. Therefore, this work considers an overriding TAGE predictor~\cite{override} that produces a base prediction in one cycle while overriding that decision with a better prediction in the second cycle if necessary. Perceptron and TAGE both require large tables. Accordingly, this work investigates how their accuracy and latency vary with the amount of hardware resources they are allowed to use.


\section{Contributions}
\label{sec:intro:contributions}

In more detail, this work makes the following contributions:

(1)~It studies the FPGA-implementation of Branch Target Buffers (BTB), including designs that fuse the BTB and the direction predictor and shows that, contrary to ASIC implementations, it is best to avoid a BTB and instead to calculate branch target addresses on-the-fly using \textit{Full Address Calculation} (FAC). It shows that a branch target predictor consists of a FAC and a Return Address Stack (RAS)~\cite{ras} is 63.2\% more accurate than a na\"ive BTB solution.

(2)~It studies the FPGA implementation of the three most commonly used branch direction predictors: bimodal~\cite{bimodal}, gshare, and gselect~\cite{McFarling}. The analysis corroborates the results of past studies showing that gshare achieves the best accuracy among the three for practical table sizes, but also shows that unlike an ASIC implementation, frequency suffers with gshare on FPGAs. It proposes \textit{gRselect}, an FPGA-friendly gselect implementation that uses a simple indexing scheme to outperform gshare by 11.4\% in terms of overall performance.

(3)~It studies the FPGA implementation of two advanced branch predictors: the Perceptron~\cite{perceptron} and TAGE~\cite{tage} predictors. It optimizes Perceptron's maximum operating frequency by introducing (i) a complement weight table to simplify the multiplication that is otherwise necessary at prediction time, and (ii) Low Order Bit (LOB) Elimination for faster summation. It finds that TAGE is too slow for single-cycle access which negates its advantage in accuracy. Accordingly, this work proposes an overriding predictor \mbox{\textit{O-TAGE-SC}} that uses a simple base predictor to provide an initial prediction in the first cycle which can be overridden in the second cycle should TAGE disagree with relatively high confidence. \mbox{O-TAGE-SC} achieves 5.2\% better instruction throughput over gRselect.



The rest of the thesis is organized as follows. Chapter~\ref{chap:background} reviews branch prediction basics and  the branch prediction schemes considered in this work. Chapter~\ref{chap:minimal} discusses the design of a minimalistic branch predictor that uses a hardware budget on par with that of Nios~II-f. Chapter~\ref{chap:advanced} presents the study on two more advanced branch predictors: the Perceptron and TAGE predictors. Chapter~\ref{chap:eval} presents the experimental evaluation results and finally Chapter~\ref{chap:conclusion} concludes.





