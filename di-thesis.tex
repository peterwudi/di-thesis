%% SUMMARY OF FEATURES:
%%
%% All environments, commands, and options provided by the `ut-thesis'
%% class will be described below, at the point where they should appear
%% in the document.  See the file `ut-thesis.cls' for more details.
%%
%% To explicitly set the pagestyle of any blank page inserted with
%% \cleardoublepage, use one of \clearemptydoublepage,
%% \clearplaindoublepage, \clearthesisdoublepage, or
%% \clearstandarddoublepage (to use the style currently in effect).
%%
%% For single-spaced quotes or quotations, use the `longquote' and
%% `longquotation' environments.


%%%%%%%%%%%%         PREAMBLE         %%%%%%%%%%%%

%%  - Default settings format a final copy (single-sided, normal
%%    margins, one-and-a-half-spaced with single-spaced notes).
%%  - For a rough copy (double-sided, normal margins, double-spaced,
%%    with the word "DRAFT" printed at each corner of every page), use
%%    the `draft' option.
%%  - The default global line spacing can be changed with one of the
%%    options `singlespaced', `onehalfspaced', or `doublespaced'.
%%  - Footnotes and marginal notes are all single-spaced by default, but
%%    can be made to have the same spacing as the rest of the document
%%    by using the option `standardspacednotes'.
%%  - The size of the margins can be changed with one of the options:
%%     . `narrowmargins' `(1 1/4" left, 3/4" others),
%%     . `normalmargins' (1 1/4" left, 1" others),
%%     . `widemargins' (1 1/4" all),
%%     . `extrawidemargins' (1 1/2" all).
%%  - The pagestyle of "cleared" pages (empty pages inserted in
%%    two-sided documents to put the next page on the right-hand side)
%%    can be set with one of the options `cleardoublepagestyleempty',
%%    `cleardoublepagestyleplain', or `cleardoublepagestylestandard'.
%%  - Any other standard option for the `report' document class can be
%%    used to override the default or draft settings (such as `10pt',
%%    `11pt', `12pt'), and standard LaTeX packages can be used to
%%    further customize the layout and/or formatting of the document.

%% *** Add any desired options. ***
\documentclass[doublespaced]{ut-thesis}
\usepackage{graphicx}

\usepackage{hyperref}
\hypersetup{
  colorlinks   = true, %Colours links instead of ugly boxes
  urlcolor     = blue, %Colour for external hyperlinks
  linkcolor    = blue, %Colour of internal links
  citecolor   = magenta %Colour of citations
}


%% *** Add \usepackage declarations here. ***
%% The standard packages `geometry' and `setspace' are already loaded by
%% `ut-thesis' -- see their documentation for details of the features
%% they provide.  In particular, you may use the \geometry command here
%% to adjust the margins if none of the ut-thesis options are suitable
%% (see the `geometry' package for details).  You may also use the
%% \setstretch command to set the line spacing to a value other than
%% single, one-and-a-half, or double spaced (see the `setspace' package
%% for details).

\newcommand{\mytilde}{{\raise.17ex\hbox{$\scriptstyle\sim$}}}

\newcommand{\kfig}[4]{ % params: file, label, caption
        \begin{figure}[!ht]
        \centering
        \includegraphics[#4]{Figures/#1}
        \vspace{-1mm}
        \caption{#3}
        \label{#2}
        \end{figure}
}



%%%%%%%%%%%%%%%%%%%%%%%%%%%%%%%%%%%%%%%%%%%%%%%%%%%%%%%%%%%%%%%%%%%%%%%%
%%                                                                    %%
%%                   ***   I M P O R T A N T   ***                    %%
%%                                                                    %%
%%  Fill in the following fields with the required information:       %%
%%   - \degree{...}       name of the degree obtained                 %%
%%   - \department{...}   name of the graduate department             %%
%%   - \gradyear{...}     year of graduation                          %%
%%   - \author{...}       name of the author                          %%
%%   - \title{...}        title of the thesis                         %%
%%%%%%%%%%%%%%%%%%%%%%%%%%%%%%%%%%%%%%%%%%%%%%%%%%%%%%%%%%%%%%%%%%%%%%%%

%% *** Change this example to appropriate values. ***
\degree{Master of Applied Science}
\department{Electrical and Computer Engineering}
\gradyear{2014}
\author{Di Wu}
\title{High Performance Branch Predictors For Soft Processors}

%% *** NOTE ***
%% Put here all other formatting commands that belong in the preamble.
%% In particular, you should put all of your \newcommand's,
%% \newenvironment's, \newtheorem's, etc. (in other words, all the
%% global definitions that you will need throughout your thesis) in a
%% separate file and use "\input{filename}" to input it here.


%% *** Adjust the following settings as desired. ***

%% List only down to subsections in the table of contents;
%% 0=chapter, 1=section, 2=subsection, 3=subsubsection, etc.
\setcounter{tocdepth}{2}

%% Make each page fill up the entire page.
\flushbottom


%%%%%%%%%%%%      MAIN  DOCUMENT      %%%%%%%%%%%%

\begin{document}

%% This sets the page style and numbering for preliminary sections.
\begin{preliminary}


%% This generates the title page from the information given above.
\maketitle

%% There should be NOTHING between the title page and abstract.
%% However, if your document is two-sided and you want the abstract
%% _not_ to appear on the back of the title page, then uncomment the
%% following line.
%\cleardoublepage

%% This generates the abstract page, with the line spacing adjusted
%% according to SGS guidelines.
\begin{abstract}
%% *** Put your Abstract here. ***
%% (At most 150 words for M.Sc. or 350 words for Ph.D.)
Branch prediction has been extensively studied in the context of application specific custom logic (ASIC) implementations. Since the tradeoffs are different for reconfigurable logic, na\"ively porting ASIC-based branch predictors to FPGAs may prove slow and/or resource-inefficient. Accordingly, this work studies the FPGA implementation of several commonly used branch predictor designs and does so in the context of simple pipelined processors, the most commonly used general purpose soft processor architecture due to its excellent balance of performance and resource cost. For this purpose, it assumes a pipelined processor implementation representative of Altera's Nios~II-f~\cite{niosii} and investigates the performance and resource cost of various branch predictors. The analysis confirms that existing designs are not efficient nor high-performing on reconfigurable logic. Accordingly, this work proposes FPGA-specific modifications that improve accuracy, resource cost, or both. Finally, this work proposes a minimalistic branch predictor \textit{gRselect} with the same hardware budget as the branch predictor in Nios~II-f, as well as an overriding TAGE~\cite{override}\cite{tage} predictor \mbox{\textit{O-TAGE-SC}} that delivers the highest performance among all the considered branch prediction schemes. 

\end{abstract}

%% Anything placed between the abstract and table of contents will
%% appear on a separate page since the abstract ends with \newpage and
%% the table of contents starts with \clearpage.  Use \cleardoublepage
%% for anything that you want to appear on a right-hand page.

%% This generates a "dedication" section, if needed
%% (uncomment to have it appear in the document).
%\begin{dedication}
%% *** Put your Dedication here. ***
%\end{dedication}

%% The `dedication' and `acknowledgements' sections do not create new
%% pages so if you want the two sections to appear on separate pages,
%% you should put an explicit \newpage between them.

%% This generates an "acknowledgements" section, if needed
%% (uncomment to have it appear in the document).
\begin{acknowledgements}
%% *** Put your Acknowledgements here. ***
I have been fortunate enough to receive enormous help and support while completing this thesis. Firstly, I would like to express my deepest gratitude to my supervisor, Prof. Andreas Moshovos, for his unending guidance and support. Secondly, it has been my great pleasure to work with all the members of the AENAO research group for their feedbacks throughout this work. Special thanks to Kaveh Aasaraai and Ian Katsuno for their unreserved help on FPGA technologies.

Finally, I would like to thank my parents for all the unconditioned love and support during my undergraduate and graduate studies. Moreover, a heartful thank you to my beloved girlfriend Mengyun Shen for always being there cheering me up, I cannot imagine finishing this work without your constant understanding and faith in me.
\end{acknowledgements}

%% This generates the Table of Contents (on a separate page).
\tableofcontents

%% This generates the List of Tables (on a separate page), if needed
%% (uncomment to have it appear in the document).
\listoftables

%% This generates the List of Figures (on a separate page), if needed
%% (uncomment to have it appear in the document).
\listoffigures

%% You can add commands here to generate any other material that belongs
%% in the head matter (for example, List of Plates, Index of Symbols, or
%% List of Appendices).

%% End of the preliminary sections: reset page style and numbering.
\end{preliminary}


%%%%%%%%%%%%%%%%%%%%%%%%%%%%%%%%%%%%%%%%%%%%%%%%%%%%%%%%%%%%%%%%%%%%%%%%
%%  Put your Chapters here; the easiest way to do this is to keep     %%
%%  each chapter in a separate file and `\include' all the files.     %%
%%  Each chapter file should start with "\chapter{ChapterName}".      %%
%%  Note that using `\include' instead of `\input' will make each     %%
%%  chapter start on a new page, and allow you to format only parts   %%
%%  of your thesis at a time by using `\includeonly'.                 %%
%%%%%%%%%%%%%%%%%%%%%%%%%%%%%%%%%%%%%%%%%%%%%%%%%%%%%%%%%%%%%%%%%%%%%%%%

%% *** Include chapter files here. ***
\chapter{Introduction}
\label{chap:introduction}

Field Programmable Gate Arrays (FPGAs) are increasingly being used in embedded and other systems. Such designs often employ one or more embedded microprocessors, and there is a trend to migrate these microprocessors to the FPGA platform primarily for reducing costs. While these soft processors cannot typically match the performance of hard processors, soft processors are flexible allowing designers to implement the exact number of processors desired and to customize them to efficiently fit the application's requirements.

Current commercial soft processors such as Altera's Nios~II~\cite{niosii} and Xilinx's Microblaze~\cite{microblaze} use in-order pipelines with five to six pipeline stages. These processors are often used for less computation-intensive applications such as system control tasks that are often control flow intensive. To support more compute-intensive applications, a key performance improving technique is branch prediction. Without branch prediction, a branch has to execute before the processor can fetch the instructions that follow. Branch prediction eliminates these stalls by guessing the target address of branches. Current state-of-the-art branch prediction techniques, e.g., TAGE~\cite{tage}, rely on dynamically collected information about past branch behaviour. Such techniques have been proven to be very effective even in deeply pipelined, highly speculative, high-performance custom processor designs.

Branch prediction has been extensively studied, mostly in the context of application specific custom logic (ASIC) implementations. However, na\"ively porting ASIC-based branch predictors to FPGAs results in slow and/or resource-inefficient implementations as the tradeoffs are different for reconfigurable compared to custom logic. Accordingly, this work studies the FPGA implementation of several commonly used and advanced branch predictor designs and does so in the context of simple pipelined processors, the most commonly used general purpose soft processor architecture due to its excellent balance of performance and resource cost. For this purpose, it assumes a pipelined processor implementation that is modeled loosely on Altera's highest performing soft-processor Nios~II-f and investigates the performance and resource cost of various branch predictors. The analysis confirms that existing designs are not efficient nor high-performing on reconfigurable logic. Accordingly, this work proposes FPGA-specific modifications that improve accuracy, resource cost, or both.

We use a two-pronged approach proposing two classes of predictors that are appropriate depending on the amount of resources we are willing to devote to the predictor component. Branch predictors make use of memories to store past branch prediction outcomes and their accuracy tends to improve the more memory resources they are given. Depending on the device, an FPGA may offer many or very few memory resources. Altera's Nios~II-f uses three Block RAMs (BRAMs) in total and only one BRAM is used for branch prediction~\cite{niosiif}. Therefore, each additional BRAM would represent a more than 1/3 overhead in terms of BRAM resources. Accordingly, we first consider minimalistic branch predictors that also use just one BRAM. In the second approach, we relax this restriction and consider more elaborate and accurate branch predictors and use as many BRAMs as necessary to improve accuracy.

\section{Design Goals}
\label{sec:introduction:goal}

This work aims to design branch predictors that (1)~operate at a high operating frequency while (2)~achieving high accuracy so that they improve execution performance. It proposes a minimalistic branch predictor that has a limited resource budget of one BRAM. The design of the minimalistic predictor considers the most commonly used direction predictors, bimodal~\cite{bimodal}, gshare and gselect~\cite{McFarling} (reviewed in Chapter~\ref{chap:background}). These predictors use a single lookup table and map relatively well onto a single BRAM.

This work also implements more advanced Perceptron and TAGE predictors. As Chapter~\ref{chap:advanced} will show, a single-cycle TAGE is prohibitively slow. Therefore, this work considers an overriding TAGE predictor~\cite{override} that produces a base prediction in one cycle while overriding that decision with a better prediction in the second cycle if necessary. Perceptron and TAGE both require large tables. Accordingly, this work investigates how their accuracy and latency vary with the amount of hardware resources they are allowed to use.


\section{Contributions}
\label{sec:intro:contributions}

In more detail, this work makes the following contributions:

(1)~It studies the FPGA-implementation of Branch Target Buffers (BTB), including designs that fuse the BTB and the direction predictor and shows that, contrary to ASIC implementations, it is best to avoid a BTB and instead to calculate branch target addresses on-the-fly using \textit{Full Address Calculation} (FAC). It shows that a branch target predictor consists of a FAC and a Return Address Stack (RAS)~\cite{ras} is 63.2\% more accurate than a na\"ive BTB solution.

(2)~It studies the FPGA implementation of the three most commonly used branch direction predictors: bimodal~\cite{bimodal}, gshare, and gselect~\cite{McFarling}. The analysis corroborates the results of past studies showing that gshare achieves the best accuracy among the three for practical table sizes, but also shows that unlike an ASIC implementation, frequency suffers with gshare on FPGAs. It proposes \textit{gRselect}, an FPGA-friendly gselect implementation that uses a simple indexing scheme to outperform gshare by 11.4\% in terms of overall performance.

(3)~It studies the FPGA implementation of two advanced branch predictors: the Perceptron~\cite{perceptron} and TAGE~\cite{tage} predictors. It optimizes Perceptron's maximum operating frequency by introducing (i) a complement weight table to simplify the multiplication that is otherwise necessary at prediction time, and (ii) Low Order Bit (LOB) Elimination for faster summation. It finds that TAGE is too slow for single-cycle access which negates its advantage in accuracy. Accordingly, this work proposes an overriding predictor \mbox{\textit{O-TAGE-SC}} that uses a simple base predictor to provide an initial prediction in the first cycle which can be overridden in the second cycle should TAGE disagree with relatively high confidence. \mbox{O-TAGE-SC} achieves 5.2\% better instruction throughput over gRselect.



The rest of the thesis is organized as follows. Chapter~\ref{chap:background} reviews branch prediction basics and  the branch prediction schemes considered in this work. Chapter~\ref{chap:minimal} discusses the design of a minimalistic branch predictor that uses a hardware budget on par with that of Nios~II-f. Chapter~\ref{chap:advanced} presents the study on two more advanced branch predictors: the Perceptron and TAGE predictors. Chapter~\ref{chap:eval} presents the experimental evaluation results and finally Chapter~\ref{chap:conclusion} concludes.






\chapter{Background and Goals}
\label{chap:background}
This chapter covers the background of branch prediction. Section~\ref{sec:background:whybp} introduces the importance of branch prediction. Section~\ref{sec:background:bpoverview} describes the structure of a canonical branch predictor. Sections~\ref{sec:background:target} and~\ref{sec:background:dirpred} overviews the branch prediction schemes considered in this thesis. Section~\ref{sec:background:fpga} briefly introduces Field Programmable Gate Arrays (FPGAs) and soft processors. Finally, Section~\ref{sec:background:goal} presents the goal of this thesis. 

\section{Why Branch Prediction?}
\label{sec:background:whybp}
In modern microprocessors, instruction pipelining is a key feature to improve performances. At each cycle, a new instruction is read from the memory at the location indicated by the program counter (PC). The pipeline works ideally when there is no transfer of control, i.e., PC is incremented by a constant at each cycle to fetch the next instruction. However, when there is a transfer of control (e.g., a taken branch instruction), the next instruction to fetch is unknown at this point. The processor has to stall the pipeline to wait until the branch is resolved to fetch the correct instruction, hence severely impacts performance.

Branch prediction is a technique to improve performance by guessing the branch target (i.e., the next instruction) to allow the processor to run speculatively. The processor fetches the predicted target, and compares the prediction with the actual target when the branch is resolved several cycles later. If the prediction was correct, the processor fetched the correct instructions without stalling the pipeline, as if the instruction stream were not disrupted. On the other hand, if the prediction was incorrect, the processor fetched wrong instructions, and the intermediate results related to those wrong instructions has to be squashed with an extra maintenance penalty. Therefore, prediction accuracy is the key to high performance.

\section{Branch Prediction Overview}
\label{sec:background:bpoverview}
Branch prediction has two aspects, i.e., direction prediction and target prediction. Branch direction prediction predicts whether a branch instruction will be taken or not, while branch target prediction predicts the target instruction address if the branch is taken.

Fig.~\ref{fig:bpcanonical} shows the organization of a typical branch predictor comprising: (1)~a direction predictor (DIR), (2)~a Branch Target Buffer (BTB), and (3)~a Return Address Stack (RAS).  

The predictor operates in the fetch stage where it aims to predict the program counter (PC), i.e., the memory address, of the instruction to fetch in the next cycle using the current instruction's PC and other dynamically collected information. The DIR guesses whether the branch will be taken or not. The BTB and the RAS guess the address for ``predicted as taken" branches and function returns respectively. The multiplexer at the end selects based on the branch type and the direction prediction whether the target is the fall through address (PC+4 in Nios~II), the target predicted by the BTB, or the target provided by the RAS. Since, at this point in time, the actual instruction is not available in a typical ASIC implementation, it is not directly possible to determine whether the instruction is a return, a branch, or some other instruction. Accordingly, a Selection Logic block uses either pre-decode information or a PC-based, dynamically populated lookup table to guess which target is best to use. With the latter scheme, when no entry exists in the lookup table, some default action is taken until the first time a branch is encountered. Once the branch executes, its type is stored in the lookup table where it serves to identify the branch type on subsequent encounters. This scheme is not perfectly accurate due to aliasing.
\kfig{bpcanonical.pdf}{fig:bpcanonical}{Canonical Branch Predictor.}{angle = 0, trim = 1in 1in 0.5in 0.8in, clip, width=0.6\textwidth}

\section{Branch Target Prediction}
\label{sec:background:target}
Branch Target Prediction usually requires a Branch Target Buffer (BTB), a cache-like structure that records the addresses of the branches and their target addresses. If a branch is predicted to be taken and there is also a BTB hit, then the next PC is set to be the predicted target. A BTB can be set-associative to reduce aliasing.

Another common structure used for branch target prediction is the Return Address Stack (RAS),  a stack-like structure that predicts the target address of function returns. When a call instruction executes, the return address of that call is pushed onto the RAS. When the processor executes the corresponding return instruction, RAS pops the return address and provides a prediction. The prediction is accurate as long as the RAS' size is less than the current call depth. Most modern processors have a shallow RAS because typical programs generally do not have very deep call depths.


\section{Branch Direction Prediction}
\label{sec:background:dirpred}
This section describes the branch direction predictors. This section first briefly introduces the \textit{static} prediction schemes. Then, more discussion will focus on \textit{dynamic} prediction schemes, starting with the simple schemes such as bimodal, gshare and gselect~\cite{McFarling}, to more complex schemes such as Perceptron~\cite{perceptron} and TAGE~\cite{tage}.

\subsection{Static Prediction Scheme}
\label{sec:background:dirpred:static}
Static branch direction predictors statically predicts a branch to be taken or non-taken. This strategy is based on the observation that most branches are taken. However, prior work has shown that approximately 30\% of all branches are \textit{unconditional} branches (e.g., functions calls and returns)~\cite{histogram}, which are of course taken. The remaining 70\% branches are \textit{conditional} branches, their directions are not strongly biased towards one direction than the other.

The main weakness of static branch predictors is that they cannot adapt to various applications. A static prediction scheme variation called ``backward-taken forward-not-taken" takes advantage of the observation that most backward branches are taken (e.g., the end of loops), while forward branches are often not taken. However, the improvements in accuracy is not significant. As the pipeline in modern processors becomes deeper, the misprediction penalties increases, which necessitates more accurate \textit{dynamic} branch predictors.

\subsection{Bimodal, Gshare and Gselect Predictor}
\label{sec:background:dirpred:bimodal}
Bimodal, Gshare and Gselect~\cite{McFarling} are some of the simplest dynamic direction prediction schemes. Fig.~\ref{fig:simpleDIR} shows the organization of these three predictors. Each of these predictors has a \textit{Pattern History Table} (PHT). The entries of a PHT are called \textit{2-bit saturating counters}, which are used to record the directions of the previously seen branches. The 2-bit saturating counters are incremented/decremented when their corresponding branch is taken/not taken, and the high order bit is used as direction prediction.

The PHT in bimodal is indexed by selected bits of PC. Since not all bits of PC are used to index the PHT, multiple different branches can be assigned to the same 2-bit saturating counter. This \textit{aliasing} can be destructive to the direction records, hence degrades performance.

Gshare and gselect utilize a structure called a \textit{Global History Register} (GHR). The GHR is a shift register that stores recently resolved branch directions. The only difference between bimodal, gshare and gselect is the indexing of the PHTs. Bimodal uses partial PC, gshare uses partial PC XORed with GHR, and gselect uses partial PC concatenated with GHR. Gshare and gselect has better accuracy over bimodal because it reduces aliasing by correlating local history (i.e., PC) with global history (i.e., GHR).
\kfig{simpleDIR.pdf}{fig:simpleDIR}{Bimodal, Gshare and Gselect.}{angle = 0, trim = 0.6in 1.1in 0.5in 1.3in, clip, width=0.6\textwidth}


\subsection{Perceptron Predictor}
\label{sec:background:dirpred:perceptron}
The Perceptron predictor uses vectors of signed weights (i.e., perceptrons) to represent correlations among branch instructions~\cite{perceptron}. Fig.~\ref{fig:perceptron} shows the structure of a Perceptron predictor. It produces a prediction through the following steps: (1)~A \textit{perceptron} is read from the table. (2)~The weights are multiplied with factors chosen based on the the corresponding global history bits. The weights are multiplied by 1 for taken and -1 for not-taken. (3)~The resulting products are summed up and a prediction is made based on the sign of the result; predict taken if the sum is positive, and not-taken otherwise. Formally, for a Perceptron predictor using $h$ history bits, let $G_i$, where $i = 1...h$, be 1 for taken and -1 for not-taken, each weight vector has $h$ weights $w_{0...h}$, where the bias constant $w_0 = 1$. The predictor has to calculate $y = w_0 + \sum_{i=1}^{h} G_iw_i$, and predict taken if $y$ is positive and not-taken otherwise.
\kfig{perceptron.pdf}{fig:perceptron}{The Perceptron branch predictor.}{angle = 0, trim = 1in 3.5in 2in 1in, clip, width=0.6\textwidth}

When a branch is resolved, the corresponding perceptron is updated. Let $y$ be the computed results at prediction time, $t$ be the branch outcome, $x_i$ be the $i_{th}$ global history, and let $\theta$ be the \textit{threshold}, the following algorithm is used to train perceptron:\vspace{8 mm}

\begin{spacing}{1} 
\texttt{if sign(y) $\neq\ t$ or $\left| y \right| \leq \theta$}

\texttt{\hspace{8mm} for i = 0 to n do}

\texttt{\hspace{16mm} $w_i = w_i + tx_i$}

\texttt{\hspace{8mm} end for}

\texttt{end if}
\end{spacing}
\vspace{8 mm}

In the above algorithm, $t$ and $x_i$ are bipolar, i.e., each of them is $1$ if the branch outcome is taken and $-1$ otherwise. The algorithm increments the $i_{th}$ weight when the outcome agrees with $x_i$, and decrements the weight otherwise. When there is a strong correlation, the absolute value of the weight will be large, which will have a large influence on the outcome. On the other hand, when there is a weak correlation, the absolute value of the weight will be close to zero, hence contributes little to the results.


\subsection{Tagged Geometric History Length Branch Predictor (TAGE)}
\label{sec:background:dirpred:tage}
The TAGE predictor features a bimodal predictor as a base predictor $T_0$  and a set of $M$ tagged predictor components $T_i$~\cite{tage}. These tagged predictor components $T_i$, where $1\leq i\leq M$, are indexed with hash functions of the branch address and the global branch/path history of various lengths. The global history lengths used for computing the indexing functions for tables $T_i$ form a geometric series, i.e., $L(i) = (int)(\alpha^{i{}^{\_}1}\times L(1)+0.5)$. TAGE achieves its high accuracy by utilizing very long history lengths judiciously.
Essentially, the base predictor captures the bulk of branches that tend to be biased, while the remaining components capture exceptions by recording specific history events that lead to exceptions that foil the base predictor.
Fig.~\ref{fig:tage} shows a 5-component TAGE predictor. Each table entry has a 3-bit saturating counter \textit{ctr} for the prediction result, a \textit{tag}, and a 2-bit useful counter \textit{u}. The table indices are produced by hashing the PC and the global history using different lengths per table $L(i)$. All tables are accessed in parallel and each table provides a valid prediction only on a tag match and provided that the corresponding useful counter is saturated. The final prediction comes from the matching tagged predictor component that uses the longest history.
\kfig{tage.pdf}{fig:tage}{A 5-component TAGE branch predictor.}{angle = 0, trim = 0.6in 0.6in 0.4in 0.2in, clip, width=0.9\textwidth}

Seznec et al. defines the \textit{provider component} as the predictor component that provides the prediction, and the alternate prediction \textit{altpred} as the prediction that would have occurred without the provider component~\cite{tage}.

When updating TAGE, the useful counter $u$ of the provider component is updated when the alternate prediction disagrees with the provider. $u$ is incremented if the provider is correct, and decremented otherwise.

The prediction counter $ctr$ of the provider is incremented if the prediction is correct and decremented otherwise. If the prediction from the provider $T_i$, where $i < M$, is incorrect, a new entry will be allocated on a predictor component $T_k$ that uses a longer history, i.e., $i<k\leq M$.


\section{Field Programmable Gate Arrays and Soft Processors}
\label{sec:background:fpga}
Field Programmable Gate Arrays (FPGAs) are chips that can be reconfigured by designers. Modern FPGAs such as Altera's Stratix IV~\cite{StratixIV} have large number of logic blocks connected by reconfigurable interconnects. The logic blocks usually consist of Look-up Tables (LUTs), registers and Block RAMs (BRAMs) etc. Designers program digital circuits with Hardware Description Languages (HDLs) such as Verilog and VHDL, and then use CAD tools provided by the FPGA vendors to synthesize the design.

FPGAs can be reconfigured to fit specific requirements of various applications, therefore they are popular choices for hardware acceleration. Due to the increasing cost and time of designing ASIC, an increasing number of embedded systems are being built using FPGA platforms~\cite{softprocessor}. These systems usually contain one or more embedded processors, necessitates high-performance FPGAs-based \textit{soft processors}. 

Since an FPGA platform is significantly different than ASIC, hard and soft processors have distinct design tradeoffs. Commercial soft processors such as Alteras's Nios~II-f~\cite{niosiif} and Xilinx's MicroBlaze~\cite{microblaze} usually have shallow pipelines, mostly because the relative speed gap between memory and logic is much smaller than on ASIC. Accordingly, branch predictor designs for soft processors also have to be re-evaluated.


\section{Design Goals}
\label{sec:background:goal}

This work aims to design branch predictors that (1)~operate at a high operating frequency while (2)~achieving high accuracy so that they improve execution performance.

This work proposes a minimalistic branch predictor for Altera's highest performing soft-processor Nios~II-f. Since Nios~II-f uses three BRAMs in total and only one BRAM is used for branch prediction~\cite{niosiif}, the proposed minimalistic branch predictor has a limited resource budget of one BRAM; each additional BRAM would represent a more than 1/3 overhead in terms of BRAM resources. The design of the minimalistic predictor considers the most commonly used direction predictors, bimodal, gshare and gselect. These predictors use a single lookup table and map relatively well onto a single BRAM.

This work also implements more advanced Perceptron and TAGE predictors. As Chapter~\ref{chap:advanced} will show, a single-cycle TAGE is prohibitively slow. Therefore, this work considers an overriding TAGE predictor~\cite{override} that produces a base prediction in one cycle while overriding that decision with a better prediction in the second cycle if necessary. Perceptron and TAGE both require large tables. Accordingly, this work investigates how their accuracy and latency vary with the amount of hardware resources they are allowed to use.



\chapter{The Minimalistic FPGA-Friendly Branch Predictors}
\label{chap:minimal}
This chapter discusses the architecture of the various minimalistic FPGA-friendly branch predictors designs. It proposes \textit{gRselect} as the best performing predictor. Section~\ref{sec:min:target} discusses target prediction, while Section~\ref{sec:min:direction} discusses direction prediction.

\section{Target Prediction}
\label{sec:min:target}
A conventional method to predict branch targets is to use Branch Target Buffers (BTB). A BTB is a table that caches branch target addresses. When a branch executes for the first time, the BTB stores the target address so that it can be used on subsequent encounters of the branch. Ideally, the BTB would be large enough so that each branch can use a separate entry. In a practical implementation, however, aliasing will occur reducing prediction accuracy.  

The simplest BTB design does not use an address tag per entry and directly predicts the target address for all instructions. Not using a tag results in a fast design that uses one direct SRAM lookup. In addition, not filtering non-branch instructions is desirable since at the time of access the instruction opcode is not available. Unfortunately, as Section~\ref{sec:eval:target} shows, when all instructions use the BTB, high destructive aliasing results in poor accuracy. To reduce aliasing, a small decode logic can prevent non-branches from updating the BTB. However, as Section~\ref{sec:eval:min:fmax} shows, while this solution increases target prediction accuracy by 30\%, the additional logic reduces the maximum frequency by 35\%. An alternative is to calculate the target address during the fetch cycle.


\subsection{Target Address Pre-calculation}
\label{sec:min:target:addrprecalc}
ASIC processor implementations use a BTB since the cache latency dominates the clock cycle leaving no room for further action. This is not true in an FPGA implementation where memory is generally faster than logic. This creates an opportunity to pre-calculate the target address for branches and thus to eliminate the BTB. In this scheme the processor fetches the instruction from the cache and then, during the same cycle, calculates the instruction's taken address. As an added benefit, address pre-calculation may improve accuracy since, if possible, it is always correct. Unfortunately, it is not possible to pre-calculate the target address for all branches. The Nios~II ISA includes two types of branches: \textit{direct} and \textit{indirect}. The target of a direct branch can be calculated using the current PC and an offset that is embedded in the instruction. Indirect branch targets are read from the register file.  

This work proposes enhancing target prediction with \textit{Full Address Calculation} (FAC), which as Fig.~\ref{fig:bimodalCP} shows, calculates the target address for all direct branches and uses a BTB\ or some other storage for indirect branches. A selection logic identifies direct branches which can benefit from FAC. FAC selects among four possible addresses depending on the branch type. The Nios~II ISA supports two schemes for direct branch target addresses, one uses a 16-bit offset (IMM16) and the other a 26-bit range (IMM26). Combined with the fall-through address (i.e., PC + 4) and the predicted address coming from the BTB, BTB+FAC uses a four-way multiplexer to select among these four possible addresses. Unfortunately, this multiplexer falls into the critical path. 
\kfig{bimodalCP.pdf}{fig:bimodalCP}{BTB with Full Address Calculation.}{angle = 0, trim = 0in 0.8in 0in 0.4in, clip, width=0.8\textwidth}

A lower cost and faster alternative to FAC is \textit{Partial Address Calculation} (PAC) which relies on typical program behavior to reduce the number of choices for the final address multiplexer. Fig.~\ref{fig:bptype} reports the relative frequency of the various branch types (see Section~\ref{sec:eval:methodology} for the methodology). Since IMM26 branches are far less frequent than IMM16 branches, PAC precalculates IMM16 branches and uses the BTB for IMM26 and indirect branches.
\kfig{histogram.pdf}{fig:bptype}{Branch Target Type Distribution.}{angle = 0, trim = 0.8in 3.5in 0.7in 3.0in, clip, width=0.8\textwidth}


\subsection{Return Address Stack}
\label{sec:min:target:ras}
The RAS is a hardware, stack-like structure that accurately predicts the target address of function returns~\cite{ras}. When a call instruction executes, it also pushes its return address  onto the RAS. Upon fetching a return instruction, the branch predictor can pop the top value from the RAS  accurately predicting the return address. As long as the RAS has enough entries, it will accurately predict all return instructions. Since the call depth of typical workloads is not deep, virtually all high performance processors incorporate a shallow RAS. For the workloads studied a 16-entry RAS proves sufficient. The RAS for a simple pipeline is simple to implement on an FPGA. Deeper pipelines may require support for speculative RAS insertions and deletions complicating its design. 
\kfig{histogram_ind_jmp.pdf}{fig:bptypeInd}{Indirect Branch Instruction Type Distribution.}{angle = 0, trim = 0.8in 3.6in 0.6in 4.1in, clip, width=0.7\textwidth}


\subsection{Eliminating the BTB}
\label{sec:min:target:nobtb}
Fig.~\ref{fig:bptypeInd} shows that for the workloads studied in this work, 97\% of the indirect branches are returns (other workloads, e.g., those using virtual functions, may behave differently). Once a RAS is included along with FAC, the BTB ends up being used for only less than 1\% of all branches. Accordingly, the BTB can be eliminated, and instead use a static, not-taken predictor for all indirect branches other than returns. Section~\ref{sec:eval:target} shows that removing the BTB in the presence of RAS reduces accuracy negligibly. Section~\ref{sec:min:fpga:nobtb} explains that lower-level FPGA related considerations also favor eliminating the BTB.

\section{Direction Prediction}
\label{sec:min:direction}
As discussed in Section~\ref{sec:background:dirpred:bimodal}, a bimodal branch direction predictor is a table of two-bit saturating counters that is indexed with a portion of the PC~\cite{bimodal}. The counters are updated up or down depending on whether the branch is taken or not respectively. The lower bit provides hysteresis to changes while the upper bit provides the prediction. As Section~\ref{sec:eval:min:direction} corroborates, increasing the number of bimodal entries does not proportionally improve accuracy. Eventually, using a larger bimodal ceases to provide any improvement since bimodal is fundamentally limited on the branch prediction patterns it can predict.

Gshare is a pattern-based predictor which uses PC \textit{XORed} with a global direction history register (GHR) to index the counter table~\cite{McFarling}. GHR stores the direction of the last few branches in a bit vector. Each GHR bit stores the direction (taken or not) of a previous branch. Combining local (i.e., PC) and global information helps to better identify the current branch. Section~\ref{sec:eval:min:direction} shows that gshare is far more accurate than bimodal. However, as Section~\ref{sec:min:fpga} explains, latency suffers with gshare due to its more complex indexing scheme.

Gselect, an alternative to gshare, indexes the counter table using a simple \textit{concatenation} of the PC and the GHR~\cite{McFarling}. Fig.~\ref{fig:gselect} shows an example of a gselect that uses eight bits of PC and eight bits of GHR (i.e., a gselect 8/8). Effectively, the bits from the PC divide the counter table into regions, and the bits from GHR select the corresponding entry within each region. However, McFarling et al. have shown that global history information weakly identifies the current branch~\cite{McFarling}. This observation suggests that a lot of the entries within each region will be underutilized, reducing the effective size of the counter table of gselect. Therefore, for practical table sizes, gshare proves more accurate than gselect. Section~\ref{sec:min:fpga:indexing} explains that with proper modification, gselect proves faster than gshare on an FPGA while sacrificing little in accuracy.
\kfig{gselect.pdf}{fig:gselect}{An example of gselect 8/8.}{angle = 0, trim = 2in 0in 2in 0in, clip, width=0.5\textwidth}

\section{FPGA implementation optimizations}
\label{sec:min:fpga}
This section discusses additional FPGA-specific implementation optimizations. Specifically, Section~\ref{sec:min:fpga:nobtb} discusses fusing the BTB and the direction predictor over a single BRAM and explains that it is best to avoid the BTB altogether. Section~\ref{sec:min:fpga:indexing} explains how gselect can be adapted to map well on an FPGA. Finally, Section~\ref{sec:min:fpga:predecode} discusses the use of pre-decoding. While this section assumes a modern, Altera FPGA, the optimizations presented should be broadly applicable.

\subsection{Eliminating the BTB}
\label{sec:min:fpga:nobtb}
As Section~\ref{sec:introduction:goal} explained, the minimalistic branch predictor aims to use one M9K BRAM. An M9K BRAM can be configured as wide as 36 bits with 256 rows~\cite{StratixIVM9K}, and it can be used to implement a fused BTB and direction predictor. Specifically, each BRAM row can store one BTB entry along with up to three direction prediction entries for a total of 768 direction entries and 256 target entries, as shown in Fig.~\ref{fig:fusedbtb}.
\kfig{fusedbtb.pdf}{fig:fusedbtb}{Fused BTB and Direction Predictor.}{trim = 2.5in 1in 1.4in 0.8in, clip, width=0.6\textwidth}

Fig.~\ref{fig:fusedbtbusage} shows the BRAM configuration of a BTB fused with direction predictors. The PC indexes a row which contains a single target prediction entry and up to three direction prediction entries. Fig.~\ref{fig:fusedbtbusage}~(a) shows a BTB fused with a 256-entry bimodal, where only one direction prediction entry is used. In this case, the last two columns of direction prediction entries are wasted. To utilize the entire BRAM, all three direction entries can be read from the BRAM, and another portion of the PC selects one of these direction prediction entries, as shown in Fig.~\ref{fig:fusedbtbusage}~(b). This is essentially a BTB fused with a 768-entry bimodal, although the larger capacity improves its accuracy, the extra multiplexer that selects between the three entries slows down the branch predictor. Fig.~\ref{fig:fusedbtbusage}~(c) shows a configuration very similar to the previous case, except that it uses two bits from the GHR to select between the three direction prediction entries. This configuration is essentially a BTB fused with a 768-entry gselect, which is much more accurate than a 768-entry bimodal because it incorporate global history. 

Unfortunately, it is not possible to use one BRAM for both a BTB and the most accurate of the direction predictors considered, gshare. The reason is that the indexing of BTB is different than gshare: BTB uses PC, while gshare uses PC XORed with GHR. It requires two read ports on a BRAM to access BTB and gshare in the same cycle. However, there are only two ports per BRAM~\cite{StratixIVM9K}. One of the two ports has to be used for updating the predictor, which leaves only one port for accessing the predictor. Hence, implementing a BTB fused with gshare is impossible, as shown in Fig.~\ref{fig:fusedbtbusage}~(d).

\kfig{fusedbtbusage.pdf}{fig:fusedbtbusage}{Fused BTB and Direction Predictor Usage.}{trim = 0.2in 0.2in 0.6in 0.2in, clip, width=1\textwidth}

As Section~\ref{sec:min:fpga:nobtb} discussed, the BTB can be eliminated when address precalculation and a RAS are used. Eliminating the BTB frees up the entire BRAM for direction prediction. The target is used only from taken branches. 


\subsection{FPGA-Friendly Direction Predictor Indexing}
\label{sec:min:fpga:indexing}
This section investigates which of the three branch direction predictors is best to use on an FPGA. As Section~\ref{sec:eval:min:direction} will show, focusing just on accuracy gshare would be the best. However, performance is not the highest with gshare since its indexing scheme results in low clock frequency. Fig.~\ref{fig:bimodalCP} depicts why the predictor's indexing scheme, when implemented on an FPGA, falls into the critical path. At every clock cycle the predicted PC is used to index the direction predictor table for the next instruction. Since BRAMs are synchronous, their index must arrive before the clock edge and thus it cannot be a registered signal of the predicted PC~\cite{StratixIVM9K}. Moreover, the setup time for the BRAM is longer than that of simple registers. Therefore, the entire path starting from the BRAM data output, through the prediction logic and back into the lookup address of the BRAM forms the critical path. This is especially a concern with gshare that uses the exclusive-or of the global history register (GHR) with Predicted PC to index the BRAM. This extra XOR logic prolongs the critical path, reducing the operating frequency.
\kfig{GSelectCP.pdf}{fig:GSelectCP}{FAC with gRselect.}{trim = 0in 5.9in 4in 0.51in, clip, width=0.8\textwidth}

Contrary to gshare, gselect has a simpler indexing scheme. Specifically, gselect uses a simple \textit{concatenation} of the GHR with Predicted PC as index. Not only is gselect's indexing fast, but it can also  be tailored to map well onto FPGAs. This work proposes \textit{gRselect} which breaks the BRAM-to-BRAM critical path by breaking the BRAM access into two parts, the first of which does not need to be ``predicted PC". It uses GHR, which is a registered signal, to index the BRAM to retrieve one wide row of direction prediction entries, and then uses the predicted PC to select between these entries. Fig.~\ref{fig:GSelectCP} shows the gRselect scheme in more detail.


\subsection{Instruction Decoding}
\label{sec:min:fpga:predecode}
To be able to select the appropriate target, the predictor needs to determine whether an instruction is a branch and if so, what kind of branch it is. This information is needed to select the corresponding predicted target through the output multiplexer. However, the decode logic lies in the critical path. To eliminate this delay, the predictor pre-decodes the instructions prior to installing them in the instruction cache. The pre-decode information is stored along with the instruction. This is similar to  typical ASIC implementations.

 




\chapter{The Advanced FPGA-Friendly Branch Predictors}
\label{chap:advanced}

Chapter~\ref{chap:minimal} proposes gRselect as the best performing minimalistic design that has a hardware budget limited to one M9K BRAM. In this chapter, we relax the constraint on hardware budget and consider more advanced and accurate branch prediction technologies to achieve better performance.

\section{Perceptron Implementation}
\label{sec:advanced:perceptron}
Section~\ref{sec:background:dirpred:perceptron} explained that the Perceptron predictor maintains vectors of weights in a table and produces a prediction through three steps. Each of these steps poses difficulties to map to an FPGA substrate. The rest of this section addresses these problems.

\subsection{Perceptron Table Organization}
\label{sec:advanced:perceptron:table}
Each weight in a perceptron is typically 8-bit wide, and Perceptron predictors usually use at least a 12-bit global history~\cite{perceptron}. The depth of the table, on the other hand, tends to be relatively shallow (e.g., 64 entries for 1KB hardware budget). This requires a very wide but shallow memory, which does not map well to BRAMs on FPGAs. For example, the widest configuration of a M9K BRAM on Altera Stratix IV is 36-bit wide times 1K entries~\cite{StratixIVM9K}. If we implement the 1KB Perceptron as proposed by Jim\'enez et al.~\cite{perceptron}, which uses 96-bit wide perceptrons with 12-bit global history, it will result in a huge resource inefficiency as shown in Fig~\ref{fig:perceptronTable}. Stratix IV chips have another larger but slower and fewer M144K BRAM~\cite{StratixIVM9K}, which can be configured as wide as 72-bit times 2K entries, clearly the inefficiency problem persists and it would impact maximum operating frequency.

Since typically the Perceptron table does not require large storage space, the proposed Perceptron implementation uses MLABs as storage, which are fast fine-grain distributed memory resources. Since 50\% of all LABs can be configured as MLAB on Altera Stratix IV devices, using MLABs does not introduce routing difficulty.
\kfig{perceptronTable.pdf}{fig:perceptronTable}{Inefficient use of M9K BRAMs to implement wide but shallow perceptron tables.}{angle = 0, trim = 1in 2in 3.4in 0.6in, clip, width=0.5\textwidth}


\subsection{Multiplication}
\label{sec:advanced:perceptron:mult}
The multiplication stage calculates the products of weights in a perceptron and their global direction histories. Since the value of the global direction history can only be either 1 or -1, the ``multiplication'' degenerates to two cases, i.e., each product can either be the true form or the 2's complement (i.e., negative) form of each weight. A straightforward implementation calculates the negative of each weight and uses a mux to select, using the corresponding global history bit, the appropriate result, as Fig.~\ref{fig:perceptronMult}(a) shows. To improve operating frequency, when updating the perceptron in the execution stage where the branch is resolved, both positive and negative forms of the updated weight can be pre-calculated, and the negatives can be stored on a complement perceptron table. This way, the multiplication stage at prediction time requires only a 2-to-1 mux, as Fig.~\ref{fig:perceptronMult}(b) shows. This optimization trade offs increased resources (it requires extra storage for the negative weights) for improved speed.
\kfig{perceptronMult.pdf}{fig:perceptronMult}{Perceptron multiplication implementation.}{angle = 0, trim = 0.3in 2.2in 3in 0.6in, clip, width=0.7\textwidth}

\subsection{Adder Tree}
\label{sec:advanced:perceptron:adder}
The adder tree sums the products from the multiplication stage. As Section~\ref{sec:eval:advanced:accuracy} will show, a global history of at least 16 bits has to be used to achieve sufficient accuracy. Implementing a 16-to-1 adder tree for 8-bit integers na\"ively degrades maximum frequency severely. The maximum frequency has to be improved for Perceptron to be practical.

This work employs \textit{Low Order Bit (LOB) Elimination} that was proposed by Aasaraai et al.~\cite{lob}. LOB\ elimination ignores the Low Order Bits (LOBs) of each weight and only use the High Order Bits (HOBs) during prediction, while still using all the bits for updates. Section~\ref{sec:eval:advanced:accuracy} shows that eliminating five LOB bits reduces accuracy by less than 1\% compared to using all eight bits, but summing fewer bits results in 14.6\% higher maximum frequency. Section~\ref{sec:eval:advanced:perf} will show that using three HOB for prediction achieves the best overall performance.

Cadenas et al.~\cite{perceptronRearrange} proposed a method to rearrange the weights stored in the table in order to reduce the number of layers of the adder tree. Assuming a Perceptron predictor uses $h$ history bits, instead of storing $h$ weights $w_i$ where $i = 1 ... h$, a new form of weights $\widetilde{w}_i$: $\widetilde{w}_i = - w_i + w_{i+1}; \widetilde{w}_{i+1} = - w_i - w_{i+1},$ for $i = 1, 3, ..., h-1$ is used. The perceptron prediction can now be computed by $y = w_0 + \sum_{i=1}^{h/2}(-G_{2i-1})\widetilde{w}_{2i-h_{2i-1}\bigoplus h_{2i}}$, where $G_i = -1$ if $h_i = 0$, and $G_i = 1$ if $h_i = 1$. Table~\ref{tab:perceptronArrangement} gives an example of this new arrangement for $i = 1$.

\begin{table}[h]
\begin{center}
\def\arraystretch{1.5}% 
\begin{tabular} {|c c|c|}
\hline
\boldmath{$h_1$} & \boldmath{$h_2$} & \boldmath{$(-G_{2i-1})\widetilde{w}_{2i-h_{2i-1}\bigoplus h_{2i}}$}~\textbf{calculation for} $i=1$ \\ \hline
0 & 0 & $(-G_1)\widetilde{w}_{2-0} = \widetilde{w}_2 = -w_1-w_2$\\ \hline
0 & 1 & $(-G_1)\widetilde{w}_{2-1} = \widetilde{w}_1 = -w_1+w_2$\\ \hline
1 & 0 & $(-G_1)\widetilde{w}_{2-1} = -\widetilde{w}_1 = w_1-w_2$\\ \hline
1 & 1 & $(-G_1)\widetilde{w}_{2-0} = -\widetilde{w}_2 = w_1+w_2$\\ \hline
\end{tabular}
\caption{Perceptron Weight Arrangement Example ($i=1$).\label{tab:perceptronArrangement}}
\end{center}
\end{table}

This new arrangement pushes part of the calculation to the less time critical update logic of the Perceptron predictor so that only $h/2$ additions have to be performed, hence reduces the number of adders required by 50\%. However, if we look at its implementation closely, this new arrangement \textit{selects} whether the sum or the difference of original weights $w_i$ and $w_{i+1}$, where $i = 1,3,5,...h-1$, should be calculated. In another word, the $h/2$ adders saved during prediction are \textit{replaced} with the same number of multiplexers. The latency difference between 3-bit 2-to-1 muxes and 3-bit adders are negligibly small, which seems to trivialize the usefulness of this arrangement. However, as Section~\ref{sec:advanced:perceptron:structure} will show, this layer of muxes combined with the muxes shown in Fig.~\ref{fig:perceptronMult} map well on FPGAs, therefore it still proves to be beneficial to use this arrangement.

Using fast adders such as carry-lookahead adders does not help to reduce the adder tree latency. This is because that the problem is not summing a few very wide numbers, but many narrow numbers. Most of the latency comes from going through layers of adders rather than propagating the carry bits. To further improve maximum frequency, this work adapts the implementation of a Wallace Tree~\cite{wallacetree}. A Wallace tree is a hardware implementation of a digital circuit that efficiently sums the partial products when multiplying two integers, which is similar to the situation that a Perceptron predictor is facing. The Wallace tree implementation proves to be 10.5\% faster than a na\"ive binary reduction tree implementation. 

\subsection{Perceptron Predictor Structure on FPGA}
\label{sec:advanced:perceptron:structure}
Fig.~\ref{fig:perceptronStructure}(a) shows the Perceptron structure with the optimizations introduced earlier in this section. The positive and negative weights with the new arrangement are read from the weight table and the complement weight table. The history bits $h_i$ and $h_{i+1}$ are XORed to select between weights $\widetilde{w}_i$ and $\widetilde{w}_{i+1}$, where $i = 1,3,5,...h-1$. They represent either the sum or the difference of the original weights $w_i$ and $w_{i+1}$, where $i = 1,3,5,...h-1$. These weights are fed into the first layer of muxes, and either the sum or the difference is selected. Then another layer of muxes determine the signs of the sums and/or differences. The outcomes are passed to the Wallace Tree to calculated the final result.

Modern FPGAs such as Altera Stratix family~\cite{StratixIV} and Xilinx Virtex family~\cite{virtex} feature full 6-input Lookup Tables (6-LUT) that can implement any 6-input functions. The two layers of muxes shown in Fig.~\ref{fig:perceptronStructure}(a) can be combined into one layer of 4-to-1 muxes, as shown in Fig.~\ref{fig:perceptronStructure}(b). Each 4-to-1 multiplexer is a 6-input function, hence can be implemented with a single 6-LUT. Therefore, the main benefit of replacing the layer of adders with the layer of muxes is that these muxes can be combined with the other layer of muxes. This latency reduction comes free from the arrangement discussed in Section~\ref{sec:advanced:perceptron:adder}.
\kfig{perceptronStructure.pdf}{fig:perceptronStructure}{Perceptron Structure on FPGA.}{angle = 0, trim = 0in 0in 0in 0in, clip, width=0.9\textwidth}


\section{TAGE Implementation}
\label{sec:advanced:tage}
Section~\ref{sec:eval:advanced:accuracy} shows that TAGE is the most accurate amongst all the direction predictors considered in this work when they use the same hardware budget. However, TAGE uses multiple tables with tagged entries that require comparator driven logic which does not map well onto FPGAs. This work uses the same TAGE configuration as proposed by Seznec~\cite{tage}. The specific configuration parameters are summarized in Table.~\ref{tab:tageconfig}. Section~\ref{sec:eval:advanced:perf} shows that the maximum frequency slowdown of TAGE is not amortized by the resulting accuracy gains. Accordingly, TAGE is best used as an overriding predictor.

\begin{table}[h]
\begin{center}
\begin{tabular} {|c|c|}
\hline
\textbf{Structure} & \textbf{Storage} \\ \hline
$T_0$ & 20 Kbits\\
$T_1$ and $T_2$ & 12 Kbits each\\
$T_3$ and $T_4$ & 26 Kbits each\\
$T_5$ & 28 Kbits\\
$T_6$ & 30 Kbits\\
$T_7$ & 16 Kbits\\
$T_8$ and $T_9$ & 17 Kbits each\\
$T_10$ & 18 Kbits\\
$T_11$ & 9.5 Kbits\\
$T_12$ & 10 Kbits\\ \hline
\textbf{Total} & 241.5Kbits \\ \hline
\end{tabular}
\caption{TAGE configuration.\label{tab:tageconfig}}
\end{center}
\end{table}


The critical path of TAGE is as follows: (1)~It performs an elaborate PC-based hashing to generate multiple table indices one per table. (2)~It accesses the tables and in parallel compares the tags of the read entries to determine whether they match. (3)~Finally each decision has to pass through cascaded layers of multiplexers to select the longest matching prediction. Although the latency of these operations is high, the path can be easily pipelined to achieve much higher operating frequency. Based on this observation, this work explores an overriding branch predictor implementation using TAGE.

Overriding branch prediction is a technique to leverage the benefits of both fast but less accurate, and slow but more accurate predictors. This technique has been used commercially, e.g., in the Alpha EV8 microprocessors~\cite{alphaEV8}. In an overriding predictor, a faster but less accurate base predictor makes a base prediction quickly, and then a slower but more accurate predictor overrides that decision if it disagrees with the base prediction. 

In this work, the base predictor is the simple bimodal predictor included in TAGE itself, i.e., $T_0$ in Fig.~\ref{fig:tage}. The bimodal predictor provides a base prediction in the first cycle, and TAGE provides a prediction at the second cycle. Section~\ref{sec:eval:advanced:accuracy} and~\ref{sec:eval:advanced:fmax} show that the overriding TAGE outperforms all the other branch prediction schemes in terms of both accuracy and maximum frequency.



\section{Branch Target Predictor}
\label{sec:advanced:target}
Chapter~\ref{chap:minimal} has shown that when using one M9K BRAM -- a hardware budget on par with that of Nios~II-f -- eliminating the BTB and using \textit{Full Address Calculation} (FAC) together with a RAS results in better performance~\cite{grselect}. It has also shown that direct branches and returns comprise over 99.8\% of all branches. Implementing FAC with RAS can cover these branches with 100\% accuracy, therefore having a BTB to cover all branches results in negligible improvement in target prediction accuracy.

Since, this chapter investigates how branch prediction accuracy can improve when additional hardware resources are used, adding a BTB for better target prediction coverage could improve target prediction accuracy. Accordingly, we consider reintroducing a BTB. However, simulations show that accuracy is still better without a BTB. This is because when the target predictor only has FAC and RAS, it never predicts indirect branches that are not returns because it is not capable to do so. As a result, the destructive aliasing in the \textit{direction} predictor is alleviated because fewer branches are being predicted. Based on this observation, the Perceptron and TAGE predictors in this chapter also uses FAC with a 16-entry RAS as the branch target predictor.




\chapter{Evaluation}
\label{chap:eval}

This section presents the experimental evaluation of the proposed branch predictors.
Section~\ref{sec:eval:methodology} details the experimental methodology. Section~\ref{sec:eval:target} evaluates the accuracy of  target address schemes showing that using a RAS with FAC is best. Section~\ref{sec:eval:direction} compares the accuracy of various direction predictions. Section~\ref{sec:eval:fpga} reports resource usage and maximum operating frequency. Finally,  Section~\ref{sec:eval:performance} reports the overall performance showing that the predictor that combines FAC, RAS, gRselect and pre-decoding is best. 


\section{Methodology}
\label{sec:eval:methodology}

To compare the predictors this work measures: (1)~Accuracy as misses per kilo instructions (MPKI) which has been shown to correlate better with performance compared to prediction accuracy alone. Processor performance in (2)~instructions per cycle (IPC), a frequency agnostic metric, that isolates the effects of implementation, and  in (3)~instructions per second (IPS), a true measure of performance. (4)~Operating frequency, and (5)~resource usage.

Simulation measures MPKI\ and IPC using a custom, cycle-accurate, full-system Nios~II simulator. The simulator boots ucLinux~\cite{uclinux}, and runs a subset of SPEC CPU2006 integer benchmarks with reference inputs~\cite{spec2k6}.

The evaluation of the minimalistic predictor uses a baseline predictor (\textsc{Base}) with a fused BTB and bimodal, as discussed in Section~\ref{sec:min:fpga:nobtb} both with 256 entries. \textsc{Base} does not decode instructions and thus uses the BTB and the bimodal for all instructions. 

All designs were implemented in Verilog and synthesized using Quartus II 13.0 on a Stratix IV chip in order to measure their maximum clock frequency and area cost. The maximum frequency is reported as the average maximum clock frequency of five placement and routing passes with different random seeds. Area usage is reported in terms of ALUTs and BRAMs used.


\section{Target Prediction}
\label{sec:eval:target}

This section measures the accuracy of target predictors. by using the baseline direction predictor while considering combinations of BTB, PAC, FAC, RAS, and larger BTBs. Fig.~\ref{fig:target:mpki} reports the reduction in target address mispredictions when using various target prediction mechanisms compared to \textsc{Base}. Using a decode logic to filter non-branches from the BTB (BTB-256) reduces mispredictions by 30\%. However, increasing the BTB size to 512 (BTB-512) or 1024 (BTB-1024) entries does not improve accuracy noticeably. In the rest of this section, all BTB configurations except for BASE use instruction filtering.

Using PAC or FAC with a 256-entry BTB reduces mispredictions by 81\% and 90\% respectively, whereas using just FAC reduces mispredictions by 84\%. Finally, using FAC with a RAS (FAC+RAS) proves best. In conclusion, eliminating the BTB and relying instead on a RAS+FAC is best in terms of accuracy. An added benefit of RAS+FAC is that it allows for a standalone, thus larger and more flexible direction predictor. 
\kfig{target_mpki.pdf}{fig:target:mpki}{Target Address Prediction Schemes: Reduction in target address misprediction over \textsc{Base}.}{trim = 0.5in 1.6in 0.5in 1.8in, clip, width=0.8\textwidth}


\subsection{The Minimalistic FPGA-Friendly Branch predictor}
\label{sec:eval:min}

\section{Direction Prediction}
\label{sec:eval:min:direction}
Fig.~\ref{fig:direction:mpki} reports the improvement in MPKI for various direction predictors relative to \textsc{Base}. Decoding the instructions and performing prediction only for branches improves MPKI by 17\% (bimodal-256). Using a larger bimodal with 4K entries further improves MPKI by only 8\% suggesting that bimodal is fundamentally limited in the branch sequences it can predict. However, using a 256- or a 4K-entry gshare improves MPKI by 79\% and 82\% respectively. 

\kfig{direction_mpki.pdf}{fig:direction:mpki}{Direction Predictors: MPKI improvement over \textsc{Base}.}{trim = 0.8in 3.8in 0.8in 4.2in, clip, width=0.7\textwidth}

Section~\ref{sec:fpga} explained why gselect may be better to implement on an FPGA. Fig.~\ref{fig:direction:mpki} show that a conventionally indexed 4K-entry gselect results in competitive accuracy, improving \textsc{Base} by 80\%. Section~\ref{sec:eval:fpga} explained that the desired number of entries for the direction predictor is either 768  or 4K when  fused with a  BTB or not respectively. The figure shows that a conventionally indexed 4K-entry gselect improves MPKI by 80\%, while the proposed FPGA-friendly organization, gRselect, improves MPKI by 79\%.
In conclusion, gshare achieves the best accuracy with gselect and gRselect offering competitive accuracies.

\section{Area and Frequency}
\label{sec:eval:min:fmax}

Fig.~\ref{fig:Fmax} shows the maximum frequency and area utilisation for each predictor design. All configurations use one BRAM. As expected, BASE is the fastest and least expensive. Adding instruction filtering reduces fmax from 353 MHz to 287 MHz, a 18\% drop. By adding address calculation, frequency drops even further. However, removing the BTB partially recovers from this frequency drop. Finally, adding a RAS to a gRselect with pre-decoding, results in a predictor that operates at 259 MHz and that uses only 147 ALUTs.
\kfig{Fmax.pdf}{fig:Fmax}{Maximum frequency and area utilisation. (PD = pre-decoding)}{trim = 0.7in 3.5in 0.7in 3.9in, clip, width=0.9\textwidth}

\subsection{Performance}
\label{sec:eval:performance}

Fig.~\ref{fig:ipc} reports  average IPC gain compared to \textsc{Base}. The bimodal predictor results in the lowest IPC  while gselect performs almost as well as gshare.
\kfig{ipc.pdf}{fig:ipc}{Improvement in IPC over \textsc{Base}.}{trim = 0.5in 1.8in 0.5in 1.8in, clip, width=0.7\textwidth}


IPC is proportional to performance only when the clock frequency remains the same. Actual performance depends  on IPS, the product of IPC and clock frequency. Fig.~\ref{fig:ips} reports overall performance in IPS. This experiment assumes a 250MHz maximum clock speed for the processor, the maximum clock frequency of Nios-II-f on the Stratix~IV~\cite{niosfmax}. The best performing predictor is a 4K-entry gRselect with FAC+RAS, no BTB, and that uses pre-decoded instructions.
\kfig{ips.pdf}{fig:ips}{IPS comparison of processors with various predictors.}{trim = 0.8in 4in 0.8in 4.2in, clip, width=0.7\textwidth}



\subsection{Advanced Branch predictors}
\label{sec:eval:advanced}











\chapter{Conclusions}
\label{chap:conclusion}

This thesis studied the implementation of branch predictors for general purpose soft processors. It targeted high frequency and high accuracy branch predictors for pipelined processors, and explored various branch predictor designs. These designs were combinations of a branch target buffer, a return address stack, commonly used simple and advanced direction predictors, pre-decoding, in-fetch instruction decoding, and target address calculation. Several FPGA-specific optimizations were proposed resulting in a branch predictor that is FPGA-friendly in that it offers high accuracy and high operating frequency.

In summary, this thesis makes the following contributions:

(1)~This thesis explores the tradeoffs of various combinations of structures for branch target prediction on FPGAs. It has shown that eliminating the BTB and using the combination of FAC and RAS as the branch target predictor is the best design for both the minimalistic predictor and the more advanced \mbox{O-TAGE-SC}.

(2)~This thesis studies the FPGA implementations of bimodal, gshare and gselect. It identifies the critical paths of these predictors. Accordingly, this thesis proposes a FPGA-friendly predictor gRselect that reverses the order of indexing to improve maximum operating frequency. It has shown that within the same hardware budget, gRselect together with FAC+RAS is the best performing branch predictor.

(3)~This thesis shows that scaling bimodal, gshare and gRselect up to 32KB improves IPC marginally, but slows down the predictor significantly. The best performing bimodal, gshare and gRselect configurations all uses 1KB hardware budget.

(4)~This thesis studies the FPGA implementations of the state-of-the-art Perceptron and TAGE branch direction predictors. Several FPGA implementation optimization techniques were proposed to achieve high operating frequency. It explored the designs tradeoffs of Perceptron and TAGE, and proposed \mbox{O-TAGE-SC}, an overriding predictor that delivers 5.2\% better instruction throughput over the 1KB gRselect.

Because an FPGA is a very different substrate than an ASIC, the design tradeoffs must be re-evaluated when designing branch predictors for soft processors. Not only the structures of the branch predictors must map well onto FPGAs, techniques that are impractical on an ASIC should also be considered.

Although \mbox{O-TAGE-SC} is \mytilde 3x more accurate than the 1KB gRselect, the IPS improvement is much smaller due to the processor's simple in-order pipeline. Considering more accurate branch predictors such as the ISL-TAGE for Nios~II would only improve IPS marginally. This work recommends the 1KB FAC+RAS with gRselect for Nios~II-f because the 5.2\% improvement does not justify the 32x more storage used. Future work may consider investigating the benefits of implementing \mbox{O-TAGE-SC} for more elaborate soft processors, e.g., an Out-of-Order soft processor, which requires a more accurate branch predictor.



%\includeonly{chapter1, chapter2, chapter3}

%% This adds a line for the Bibliography in the Table of Contents.
\addcontentsline{toc}{chapter}{Bibliography}
%% *** Set the bibliography style. ***
%% (change according to your preference/requirements)
\bibliographystyle{plain}
%% *** Set the bibliography file. ***
%% ("thesis.bib" by default; change as needed)
\bibliography{thesis}

%% *** NOTE ***
%% If you don't use bibliography files, comment out the previous line
%% and use \begin{thebibliography}...\end{thebibliography}.  (In that
%% case, you should probably put the bibliography in a separate file and
%% `\include' or `\input' it here).

\end{document}
