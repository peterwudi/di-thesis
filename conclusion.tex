\chapter{Conclusion}
\label{chap:conclusion}


This work studied the implementation of branch predictors for general purpose soft processors.  This work targeted high frequency, low area overhead branch predictors for pipelined processors, and explored various branch predictor designs. These designs were combinations of a branch target buffer, a return address stack, three commonly used direction predictors, pre-decoding, in-fetch instruction decoding, and target address calculation. Several FPGA-specific optimizations were proposed resulting in a branch predictor that is FPGA-friendly in that it offers high accuracy, high operating frequency with few resources. Future work may consider more elaborate processor designs and/or other workloads that may benefit from more elaborate branch direction predictors than the ones considered in this work.



This work studied the implementation of  Perceptron and TAGE branch predictors for general purpose soft processors. It explored the designs tradeoffs of Perceptron and TAGE, and proposed \mbox{O-TAGE-SC}, an overriding predictor that delivers 5.2\% better instruction throughput over the best performing previously proposed predictor, gRselect. Several FPGA implementation optimization techniques were proposed to achieve high operating frequency. Although \mbox{O-TAGE-SC} is \mytilde 3x more accurate than the 1KB gRselect, the IPS improvement is  smaller due to the processor's simple in-order pipeline. Future work may consider investigating the benefits of implementing \mbox{O-TAGE-SC} for more elaborated soft processors, e.g., an Out-of-Order soft processor. 
